\documentclass[]{article}
\usepackage{lmodern}
\usepackage{amssymb,amsmath}
\usepackage{ifxetex,ifluatex}
\usepackage{fixltx2e} % provides \textsubscript
\ifnum 0\ifxetex 1\fi\ifluatex 1\fi=0 % if pdftex
  \usepackage[T1]{fontenc}
  \usepackage[utf8]{inputenc}
\else % if luatex or xelatex
  \ifxetex
    \usepackage{mathspec}
  \else
    \usepackage{fontspec}
  \fi
  \defaultfontfeatures{Ligatures=TeX,Scale=MatchLowercase}
  \newcommand{\euro}{€}
\fi
% use upquote if available, for straight quotes in verbatim environments
\IfFileExists{upquote.sty}{\usepackage{upquote}}{}
% use microtype if available
\IfFileExists{microtype.sty}{%
\usepackage{microtype}
\UseMicrotypeSet[protrusion]{basicmath} % disable protrusion for tt fonts
}{}
\usepackage[margin=1in]{geometry}
\usepackage{hyperref}
\PassOptionsToPackage{usenames,dvipsnames}{color} % color is loaded by hyperref
\hypersetup{unicode=true,
            pdfborder={0 0 0},
            breaklinks=true}
\urlstyle{same}  % don't use monospace font for urls
\usepackage{color}
\usepackage{fancyvrb}
\newcommand{\VerbBar}{|}
\newcommand{\VERB}{\Verb[commandchars=\\\{\}]}
\DefineVerbatimEnvironment{Highlighting}{Verbatim}{commandchars=\\\{\}}
% Add ',fontsize=\small' for more characters per line
\usepackage{framed}
\definecolor{shadecolor}{RGB}{248,248,248}
\newenvironment{Shaded}{\begin{snugshade}}{\end{snugshade}}
\newcommand{\KeywordTok}[1]{\textcolor[rgb]{0.13,0.29,0.53}{\textbf{{#1}}}}
\newcommand{\DataTypeTok}[1]{\textcolor[rgb]{0.13,0.29,0.53}{{#1}}}
\newcommand{\DecValTok}[1]{\textcolor[rgb]{0.00,0.00,0.81}{{#1}}}
\newcommand{\BaseNTok}[1]{\textcolor[rgb]{0.00,0.00,0.81}{{#1}}}
\newcommand{\FloatTok}[1]{\textcolor[rgb]{0.00,0.00,0.81}{{#1}}}
\newcommand{\ConstantTok}[1]{\textcolor[rgb]{0.00,0.00,0.00}{{#1}}}
\newcommand{\CharTok}[1]{\textcolor[rgb]{0.31,0.60,0.02}{{#1}}}
\newcommand{\SpecialCharTok}[1]{\textcolor[rgb]{0.00,0.00,0.00}{{#1}}}
\newcommand{\StringTok}[1]{\textcolor[rgb]{0.31,0.60,0.02}{{#1}}}
\newcommand{\VerbatimStringTok}[1]{\textcolor[rgb]{0.31,0.60,0.02}{{#1}}}
\newcommand{\SpecialStringTok}[1]{\textcolor[rgb]{0.31,0.60,0.02}{{#1}}}
\newcommand{\ImportTok}[1]{{#1}}
\newcommand{\CommentTok}[1]{\textcolor[rgb]{0.56,0.35,0.01}{\textit{{#1}}}}
\newcommand{\DocumentationTok}[1]{\textcolor[rgb]{0.56,0.35,0.01}{\textbf{\textit{{#1}}}}}
\newcommand{\AnnotationTok}[1]{\textcolor[rgb]{0.56,0.35,0.01}{\textbf{\textit{{#1}}}}}
\newcommand{\CommentVarTok}[1]{\textcolor[rgb]{0.56,0.35,0.01}{\textbf{\textit{{#1}}}}}
\newcommand{\OtherTok}[1]{\textcolor[rgb]{0.56,0.35,0.01}{{#1}}}
\newcommand{\FunctionTok}[1]{\textcolor[rgb]{0.00,0.00,0.00}{{#1}}}
\newcommand{\VariableTok}[1]{\textcolor[rgb]{0.00,0.00,0.00}{{#1}}}
\newcommand{\ControlFlowTok}[1]{\textcolor[rgb]{0.13,0.29,0.53}{\textbf{{#1}}}}
\newcommand{\OperatorTok}[1]{\textcolor[rgb]{0.81,0.36,0.00}{\textbf{{#1}}}}
\newcommand{\BuiltInTok}[1]{{#1}}
\newcommand{\ExtensionTok}[1]{{#1}}
\newcommand{\PreprocessorTok}[1]{\textcolor[rgb]{0.56,0.35,0.01}{\textit{{#1}}}}
\newcommand{\AttributeTok}[1]{\textcolor[rgb]{0.77,0.63,0.00}{{#1}}}
\newcommand{\RegionMarkerTok}[1]{{#1}}
\newcommand{\InformationTok}[1]{\textcolor[rgb]{0.56,0.35,0.01}{\textbf{\textit{{#1}}}}}
\newcommand{\WarningTok}[1]{\textcolor[rgb]{0.56,0.35,0.01}{\textbf{\textit{{#1}}}}}
\newcommand{\AlertTok}[1]{\textcolor[rgb]{0.94,0.16,0.16}{{#1}}}
\newcommand{\ErrorTok}[1]{\textcolor[rgb]{0.64,0.00,0.00}{\textbf{{#1}}}}
\newcommand{\NormalTok}[1]{{#1}}
\usepackage{graphicx,grffile}
\makeatletter
\def\maxwidth{\ifdim\Gin@nat@width>\linewidth\linewidth\else\Gin@nat@width\fi}
\def\maxheight{\ifdim\Gin@nat@height>\textheight\textheight\else\Gin@nat@height\fi}
\makeatother
% Scale images if necessary, so that they will not overflow the page
% margins by default, and it is still possible to overwrite the defaults
% using explicit options in \includegraphics[width, height, ...]{}
\setkeys{Gin}{width=\maxwidth,height=\maxheight,keepaspectratio}
\setlength{\parindent}{0pt}
\setlength{\parskip}{6pt plus 2pt minus 1pt}
\setlength{\emergencystretch}{3em}  % prevent overfull lines
\providecommand{\tightlist}{%
  \setlength{\itemsep}{0pt}\setlength{\parskip}{0pt}}
\setcounter{secnumdepth}{0}

%%% Use protect on footnotes to avoid problems with footnotes in titles
\let\rmarkdownfootnote\footnote%
\def\footnote{\protect\rmarkdownfootnote}

%%% Change title format to be more compact
\usepackage{titling}

% Create subtitle command for use in maketitle
\providecommand{\subtitle}[1]{
  \posttitle{
    \begin{center}\large#1\end{center}
    }
}

\setlength{\droptitle}{-2em}

  \title{}
    \pretitle{\vspace{\droptitle}}
  \posttitle{}
    \author{}
    \preauthor{}\postauthor{}
    \date{}
    \predate{}\postdate{}
  

% Redefines (sub)paragraphs to behave more like sections
\ifx\paragraph\undefined\else
\let\oldparagraph\paragraph
\renewcommand{\paragraph}[1]{\oldparagraph{#1}\mbox{}}
\fi
\ifx\subparagraph\undefined\else
\let\oldsubparagraph\subparagraph
\renewcommand{\subparagraph}[1]{\oldsubparagraph{#1}\mbox{}}
\fi


\begin{document}

\section{Reproducible Research: Peer Assessment
1}\label{reproducible-research-peer-assessment-1}

In the following report we will present analysis of the activity
monitoring data made available for this assignment (dataset available
\href{https://d396qusza40orc.cloudfront.net/repdata\%2Fdata\%2Factivity.zip}{here}).

\subsection{Loading and preprocessing the
data}\label{loading-and-preprocessing-the-data}

\begin{enumerate}
\def\labelenumi{\arabic{enumi}.}
\tightlist
\item
  Load the data (i.e.~read.csv())
\end{enumerate}

The csv is stored as \emph{activity.csv} in the working directory.

\begin{Shaded}
\begin{Highlighting}[]
\NormalTok{d <-}\StringTok{ }\KeywordTok{read.csv}\NormalTok{(}\StringTok{"activity.csv"}\NormalTok{)}
\end{Highlighting}
\end{Shaded}

\begin{enumerate}
\def\labelenumi{\arabic{enumi}.}
\setcounter{enumi}{1}
\item
\end{enumerate}

\subsection{What is mean total number of steps taken per
day?}\label{what-is-mean-total-number-of-steps-taken-per-day}

For this part of the assignment, we ignore the missing values in the
dataset.

\begin{enumerate}
\def\labelenumi{\arabic{enumi}.}
\tightlist
\item
  Make a histogram of the total number of steps taken each day
\end{enumerate}

We first find the daily sums to generate the histogram:

\begin{Shaded}
\begin{Highlighting}[]
\NormalTok{sums_d <-}\StringTok{ }\KeywordTok{tapply}\NormalTok{(d$steps, d$date, sum)}
\end{Highlighting}
\end{Shaded}

Following is the visualization:

\begin{Shaded}
\begin{Highlighting}[]
\KeywordTok{hist}\NormalTok{(sums_d, }\DataTypeTok{xlab=}\StringTok{"Total number of steps per day"}\NormalTok{,}
     \DataTypeTok{xlim=}\KeywordTok{c}\NormalTok{(}\DecValTok{1}\NormalTok{, }\DecValTok{25000}\NormalTok{), }\DataTypeTok{ylim=}\KeywordTok{c}\NormalTok{(}\DecValTok{1}\NormalTok{, }\DecValTok{30}\NormalTok{), }\DataTypeTok{breaks=}\DecValTok{10}\NormalTok{, }\DataTypeTok{col=}\StringTok{"grey"}\NormalTok{,}
     \DataTypeTok{main=}\StringTok{"Histogram: Original Data"}\NormalTok{)}
\end{Highlighting}
\end{Shaded}

\includegraphics{PA1_template_files/figure-latex/unnamed-chunk-3-1.pdf}

\begin{enumerate}
\def\labelenumi{\arabic{enumi}.}
\setcounter{enumi}{1}
\tightlist
\item
  Calculate and report the mean and median total number of steps taken
  per day
\end{enumerate}

\begin{Shaded}
\begin{Highlighting}[]
\NormalTok{mean_d <-}\StringTok{ }\KeywordTok{mean}\NormalTok{(sums_d, }\DataTypeTok{na.rm=}\NormalTok{T)}
\NormalTok{median_d <-}\StringTok{ }\KeywordTok{median}\NormalTok{(sums_d, }\DataTypeTok{na.rm=}\NormalTok{T)}
\end{Highlighting}
\end{Shaded}

Therefore, the mean and median total number of steps taken per day are
1.0766189\times 10\^{}\{4\} and 10765, respectively.

\subsection{What is the average daily activity
pattern?}\label{what-is-the-average-daily-activity-pattern}

\begin{enumerate}
\def\labelenumi{\arabic{enumi}.}
\tightlist
\item
  Make a time series plot with the 5-minute interval (x-axis) and the
  average number of steps taken, averaged across all days (y-axis)
\end{enumerate}

We derive the interval averages:

\begin{Shaded}
\begin{Highlighting}[]
\NormalTok{ints <-}\StringTok{ }\KeywordTok{tapply}\NormalTok{(d$steps, d$interval, function(x) \{}\KeywordTok{mean}\NormalTok{(x, }\DataTypeTok{na.rm=}\NormalTok{T)\})}
\end{Highlighting}
\end{Shaded}

The we use R's built-in ts function to generate the time series graph:

\begin{Shaded}
\begin{Highlighting}[]
\KeywordTok{ts.plot}\NormalTok{(}\KeywordTok{ts}\NormalTok{(ints), }\DataTypeTok{xlab=}\StringTok{"Intervals"}\NormalTok{, }\DataTypeTok{ylab=}\StringTok{"# of Steps"}\NormalTok{, }\DataTypeTok{main=}\StringTok{"Average Daily Activity Pattern"}\NormalTok{, }\DataTypeTok{type=}\StringTok{"l"}\NormalTok{)}
\end{Highlighting}
\end{Shaded}

\includegraphics{PA1_template_files/figure-latex/unnamed-chunk-6-1.pdf}

\begin{enumerate}
\def\labelenumi{\arabic{enumi}.}
\setcounter{enumi}{1}
\tightlist
\item
  Which 5-minute interval, on average across all the days in the
  dataset, contains the maximum number of steps?
\end{enumerate}

We simply find the max value of the \emph{ints} array.

\begin{Shaded}
\begin{Highlighting}[]
\KeywordTok{which.max}\NormalTok{(ints)}
\end{Highlighting}
\end{Shaded}

\begin{verbatim}
## 835 
## 104
\end{verbatim}

Therefore, 104 is the maximum number of steps, occurring at the 5-minute
starting at the 835th minute.

\subsection{Imputing missing values}\label{imputing-missing-values}

There are a number of days/intervals in the original dataset where there
are missing values (coded as NA). The presence of missing days may
introduce bias into some calculations or summaries of the data.

\begin{enumerate}
\def\labelenumi{\arabic{enumi}.}
\tightlist
\item
  Calculate and report the total number of missing values in the dataset
  (i.e.~the total number of rows with NAs)
\end{enumerate}

\begin{Shaded}
\begin{Highlighting}[]
\KeywordTok{sum}\NormalTok{(}\KeywordTok{is.na}\NormalTok{(d$steps))}
\end{Highlighting}
\end{Shaded}

\begin{verbatim}
## [1] 2304
\end{verbatim}

\begin{enumerate}
\def\labelenumi{\arabic{enumi}.}
\setcounter{enumi}{1}
\tightlist
\item
  Devise a strategy for filling in all of the missing values in the
  dataset. The strategy does not need to be sophisticated. For example,
  you could use the mean/median for that day, or the mean for that
  5-minute interval, etc.
\end{enumerate}

We are just going to substitute the missing values with zeros.

\begin{enumerate}
\def\labelenumi{\arabic{enumi}.}
\setcounter{enumi}{2}
\tightlist
\item
  Create a new dataset that is equal to the original dataset but with
  the missing data filled in.
\end{enumerate}

We create a new dataset \emph{e}:

\begin{Shaded}
\begin{Highlighting}[]
\NormalTok{e <-}\StringTok{ }\NormalTok{d}
\NormalTok{e$steps[}\KeywordTok{is.na}\NormalTok{(e$steps)] <-}\StringTok{ }\DecValTok{0}
\end{Highlighting}
\end{Shaded}

\begin{enumerate}
\def\labelenumi{\arabic{enumi}.}
\setcounter{enumi}{3}
\tightlist
\item
  Make a histogram of the total number of steps taken each day and
  Calculate and report the mean and median total number of steps taken
  per day. Do these values differ from the estimates from the first part
  of the assignment? What is the impact of imputing missing data on the
  estimates of the total daily number of steps?
\end{enumerate}

\begin{Shaded}
\begin{Highlighting}[]
\CommentTok{#get statistics by date for new date}
\NormalTok{sums_e <-}\StringTok{ }\KeywordTok{tapply}\NormalTok{(e$steps, e$date, sum)}
\NormalTok{mean_e <-}\StringTok{ }\KeywordTok{mean}\NormalTok{(sums_e, }\DataTypeTok{na.rm=}\NormalTok{T)}
\NormalTok{median_e <-}\StringTok{ }\KeywordTok{median}\NormalTok{(sums_e, }\DataTypeTok{na.rm=}\NormalTok{T)}

\CommentTok{#plot histogram new data}
\KeywordTok{hist}\NormalTok{(sums_e, }\DataTypeTok{xlab=}\StringTok{"Total number of steps per day"}\NormalTok{, }
     \DataTypeTok{xlim=}\KeywordTok{c}\NormalTok{(}\DecValTok{1}\NormalTok{, }\DecValTok{25000}\NormalTok{), }\DataTypeTok{ylim=}\KeywordTok{c}\NormalTok{(}\DecValTok{1}\NormalTok{, }\DecValTok{30}\NormalTok{), }\DataTypeTok{breaks=}\DecValTok{10}\NormalTok{, }\DataTypeTok{col=}\StringTok{"grey"}\NormalTok{,}
     \DataTypeTok{main=}\StringTok{"Histogram: Adjusted Data"}\NormalTok{)}
\end{Highlighting}
\end{Shaded}

\includegraphics{PA1_template_files/figure-latex/unnamed-chunk-10-1.pdf}

The mean and median total number of steps taken per day for this new
dataset are 9354.2295082 and 1.0395\times 10\^{}\{4\}, respectively. As
expected, the mean is 1411.959171 lower, and the median is also 370
lower. This reflects the skewing effect the additional 0's have on the
overall dataset.

\subsection{Are there differences in activity patterns between weekdays
and
weekends?}\label{are-there-differences-in-activity-patterns-between-weekdays-and-weekends}

\begin{Shaded}
\begin{Highlighting}[]
\KeywordTok{Sys.setlocale}\NormalTok{(}\StringTok{"LC_TIME"}\NormalTok{, }\StringTok{"English"}\NormalTok{)}
\end{Highlighting}
\end{Shaded}

\begin{verbatim}
## [1] ""
\end{verbatim}

\begin{Shaded}
\begin{Highlighting}[]
\NormalTok{wd <-}\StringTok{ }\NormalTok{e$date}
\NormalTok{e <-}\StringTok{ }\KeywordTok{cbind}\NormalTok{(e, wd)}
\NormalTok{e$wd <-}\StringTok{ }\KeywordTok{weekdays}\NormalTok{(}\KeywordTok{as.Date}\NormalTok{(wd))}
\NormalTok{e[(e$wd ==}\StringTok{ "Sunday"}\NormalTok{) |}\StringTok{ }\NormalTok{(e$wd ==}\StringTok{ "Saturday"}\NormalTok{), }\DecValTok{4}\NormalTok{] <-}\StringTok{ "weekend"}
\NormalTok{e[(e$wd !=}\StringTok{ "weekend"}\NormalTok{), }\DecValTok{4}\NormalTok{] <-}\StringTok{ "weekday"}

\NormalTok{w1 <-}\StringTok{ }\KeywordTok{tapply}\NormalTok{(e[(e$wd ==}\StringTok{ "weekday"}\NormalTok{),}\DecValTok{1}\NormalTok{], e[(e$wd ==}\StringTok{ "weekday"}\NormalTok{),}\DecValTok{3}\NormalTok{], mean)}
\NormalTok{w2 <-}\StringTok{ }\KeywordTok{tapply}\NormalTok{(e[(e$wd ==}\StringTok{ "weekend"}\NormalTok{),}\DecValTok{1}\NormalTok{], e[(e$wd ==}\StringTok{ "weekend"}\NormalTok{),}\DecValTok{3}\NormalTok{], mean)}

\NormalTok{w1_df <-}\StringTok{ }\KeywordTok{as.data.frame}\NormalTok{(}\KeywordTok{as.vector}\NormalTok{(w1))}
\NormalTok{w1_df <-}\StringTok{ }\KeywordTok{cbind}\NormalTok{(w1_df, e[}\DecValTok{1}\NormalTok{:}\DecValTok{288}\NormalTok{, }\DecValTok{3}\NormalTok{])}
\NormalTok{w1_df <-}\StringTok{ }\KeywordTok{cbind}\NormalTok{(w1_df, }\KeywordTok{rep}\NormalTok{(}\StringTok{"weekday"}\NormalTok{, }\DataTypeTok{each=}\DecValTok{288}\NormalTok{))}
\KeywordTok{colnames}\NormalTok{(w1_df) <-}\StringTok{ }\KeywordTok{c}\NormalTok{(}\StringTok{"steps"}\NormalTok{, }\StringTok{"interval"}\NormalTok{, }\StringTok{"wd"}\NormalTok{)}

\NormalTok{w2_df <-}\StringTok{ }\KeywordTok{as.data.frame}\NormalTok{(}\KeywordTok{as.vector}\NormalTok{(w2))}
\NormalTok{w2_df <-}\StringTok{ }\KeywordTok{cbind}\NormalTok{(w2_df, e[}\DecValTok{1}\NormalTok{:}\DecValTok{288}\NormalTok{, }\DecValTok{3}\NormalTok{])}
\NormalTok{w2_df <-}\StringTok{ }\KeywordTok{cbind}\NormalTok{(w2_df, }\KeywordTok{rep}\NormalTok{(}\StringTok{"weekend"}\NormalTok{, }\DataTypeTok{each=}\DecValTok{288}\NormalTok{))}
\KeywordTok{colnames}\NormalTok{(w2_df) <-}\StringTok{ }\KeywordTok{c}\NormalTok{(}\StringTok{"steps"}\NormalTok{, }\StringTok{"interval"}\NormalTok{, }\StringTok{"wd"}\NormalTok{)}

\NormalTok{lat <-}\StringTok{ }\KeywordTok{rbind}\NormalTok{(w1_df, w2_df)}

\KeywordTok{library}\NormalTok{(lattice)}
\KeywordTok{xyplot}\NormalTok{(steps~interval|wd, lat,}
       \DataTypeTok{main=}\StringTok{"Activity patterns: weekdays vs. weekends"}\NormalTok{,}
       \DataTypeTok{xlab=}\StringTok{"Interval"}\NormalTok{, }\DataTypeTok{ylab=}\StringTok{"Number of steps"}\NormalTok{,}
       \DataTypeTok{layout=}\KeywordTok{c}\NormalTok{(}\DecValTok{1}\NormalTok{,}\DecValTok{2}\NormalTok{), }\DataTypeTok{type=}\StringTok{"l"}\NormalTok{)}
\end{Highlighting}
\end{Shaded}

\includegraphics{PA1_template_files/figure-latex/unnamed-chunk-12-1.pdf}

Clearly, there are noticeable differences in the weekly patterns. E.g.
on weekdays there is only 1 spike in the morning that stands out, while
there are a number of similarly tall spikes throughout the middle of the
day on weekends. However, while the weekend spikes occur more
consistently, they are still shorter than the mid-morning weekday spike.

\end{document}
